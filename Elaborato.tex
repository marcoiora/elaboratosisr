\documentclass[a4paper, 11pt]{report}

\usepackage[latin1]{inputenc}
\usepackage[italian]{babel}
\usepackage{makeidx}
\usepackage[pdfborder=0]{hyperref}
\usepackage{graphicx}
\usepackage{amsmath}

\title{\Huge\textbf{Elaborato\\di\\Sistemi Informativi e Servizi in Rete}\\ \vspace{1cm} \huge Negozio di vendita on-line di Magliette Personalizzate}
\author{\textit{Iora Marco} Matricola: 65574\and \textit{Lorenzi Roberta} Matricola: 72361\and \textit{Piccinelli Andrea} Matricola: 83392}
\date{A.A. 2011--2012}

\begin{document}
\begin{figure}[t]
	\centering
	\includegraphics[height=4cm, width=4cm]{logoUnibs.jpg}\\
	Universit� degli Studi di Brescia\\Facolt� di Ingegneria\\Corso di Laurea Magistrale e di Laurea Specialistica\\ in Ingegneria Informatica
\end{figure}
\maketitle

\tableofcontents

\chapter{Analisi dei Requisiti}
\newpage

\section{Descrizione Generale}
\subsection{Scenario}
Si vuole mettere a disposizione di potenziali clienti un \textit{negozio virtuale} in cui potranno
acquistare capi di abbigliamento (magliette), seguendoli durante tutti i passi della realizzazione 
(personalizzazione delle stampe e creazione magliette, effettuare un ordine e controllo dello stesso).\\
Con questo sito, quindi, i potenziali clienti potranno sia visualizzare il catalogo delle tipologie
di magliette possibili e delle stampe presenti, sia creare o personalizzare le proprie magliette 
secondo le proprie preferenze ed esigenze con l'invio di file scelti dall'utente, sia effettuare 
l'ordine e monitorarlo fino alla consegna.\\
Questo sito � fruito anche dal personale interno all'azienda ed in particolare agli addetti allo 
stampaggio e confezionamento delle magliette. Questi potranno sia visualizzare gli ordini effettuati dai
clienti finali e prenderli in carico, sia visualizzare gli ordini gi� presi in carico e non ancora
finiti, sia evadere l'ordine (cio� affidarlo al corriere) inserendo l'identificativo di tracciabilit�
della spedizione dato dal corriere.\\
L'azienda incarica una societ� esterna per la spedizione della merce ai propri clienti.
\subsection{Progettazione delle strutture dati e delle tipologie di utenti}
Il modello dei dati include:
\begin{itemize}
	\item l'azienda descritta da nome, indirizzo, partita iva, codice fiscale, ragione sociale, e-mail,
	descrizione, informazioni sulla spedizione (costi e tempi di spedizione)
	\item le magliette fra le quali l'utente potr� scegliere, descritte da nome, colore, modello (per
	esempio, manica lunga, manica corta, materiale), modalit� di stampaggio (per esempio, solo fronte, 
	solo retro, fronte e retro), prezzo, disponibilit�
	\item le stampe, descritte da nome, posizione della stampa (fronte e/o retro), 
	dimensioni, prezzo, disponibilit�
	\item la gestione degli ordini, descritti da data/ora dell'ordine, chi ha effettuato l'ordine, 
	data/ora di presa in carico dell'ordine, operaio che ha preso in carico l'ordine, data/ora di 
	spedizione, stato dell'ordine (evaso, cio� affidato al corriere, non evaso, cio� in stampa,
	non ancora gestito, annullato), identificatore della spedizione
\end{itemize}
Verr� inoltre realizzata la gerachia delle diverse tipologie di utenti del sito.\\
In particolare vanno distinti:
\begin{itemize}
	\item gli utenti finali non registrati al sito che potranno visualizzare le seguenti pagine:
	\begin{itemize}
		\item il catalogo dei prodotti (magliette e stampe)
		\item la pagina di presentazione dell'azienda e delle informazioni sulla spedizione (costi e
		tempi di consegna). 
		\item pagina per la registrazione
		\item form di login
	\end{itemize}
	\item gli utenti registrati che, oltre alle pagine visualizzate da un utente non registrato,
	potranno accedere ad ulteriori servizi come:
	\begin{itemize}
		\item la pagina di creazione/personalizzazione della maglietta
		\item la pagina della lista degli ordini effettuati, con informazioni e stato
		\item la pagina per l'annullamento dell'ordine (fino a quando non � ancora stato evaso, cio� in 
		stampa, da un operaio stampatore) con contatto con l'operaio stampatore che ha preso in carico
		il suo ordine (senza identificazione dell'operaio stesso)
	\end{itemize}
	\item gli operai stampatori che potranno visualizzare:
	\begin{itemize}
		\item la pagina degli ordini non ancora gestiti da cui scegliere un ordine da prendere in 
		carico (pu� avere anche pi� ordini in carico)
		\item la pagina degli ordini che ha gi� in carico
		\item la pagina con le informazioni di annullamento di un ordine con contatto con il cliente
		finale (senza identificazione del cliente stesso)
	\end{itemize}
	\item l'amministratore che dovr� gestire gli account degli utenti registrati e degli operai 
	dell'azienda. Potr� inoltre visualizzare:
	\begin{itemize}
		\item la pagina degli ordini classificati in evasi, non evasi, non ancora gestiti e annullati e con tutte 
		le informazioni (data/ora dell'ordine e chi ha effettuato l'ordine, data/ora presa in carico
		dall'operaio e dati dell'operaio, data/ora di spedizione (affidamento al corriere))
		\item registro dei contatti fra operai e clienti che hanno effettuato l'ordine con informazioni
		di eventuali annullamenti
		\item pagina di inserimento/modifica/cancellazione dei prodotti in vendita
	\end{itemize}
\end{itemize}
\subsection{Progettazione dell'ipertesto}
I visitatori accedono inizialmente alla sezione pubblica del sito, dove possono visualizzare le 
informazioni sull'azienda, sui prodotti (magliette e stampe) e sulla spedizione (tempi e costi)
muovendosi fra le varie pagine del sito.\\
Se vogliono personalizzare o ordinare i prodotti, i clienti devono registrarsi e quindi effettuare il 
login per accedere alla sezione privata del sito.\\
Gli operai stampatori inizialmente accederanno alla sezione pubblica del sito e quindi dovranno
effettuale il login per accedere alla loro sezione privata. In questa sezione si muoveranno fra le 
varie pagine del sito per poter visualizzare gli ordini e informazioni particolari.\\
In qualit� di amministratore del portale, l'azienda avr� il compito sia di gestire gli utenti registrati
e gli operai stampatori, inserendone di nuovi oppure cancellando utenti esistenti, sia inserire o 
modificare le informazioni relative ai prodotti in vendita.\\
Gli utenti registrati e gli operai stampatori possono accedere alla propria sezione privata per 
modificare i propri dati.

\newpage
\section{Gruppi di Utenti}

\newpage
\section{Gerarchia dei Gruppi di Utenti}

\newpage
\section{Schede dei Gruppi di Utenti}

\newpage
\section{Diagrammi dei Casi d'Uso}

\newpage
\section{Dizionario dei Dati}

\newpage
\section{Specifica delle Site View}

\newpage
\section{Linee guida dello Stile Grafico}

\newpage
\textbf{UTENTI}:
\begin{itemize}
	\item \textbf{Amministratore}
		\begin{itemize}
			\item gestione account operai azienda
			\item pagina degli ordini di diverso tipo:
				\begin{itemize}
					\item evasi (affidati al corriere: con ID di spedizione rilasciato dal corriere)
					\item non evasi (in stampa)
					\item non ancora gestiti
				\end{itemize}
			ogni ordine con i seguenti dati:
				\begin{itemize}
					\item data/ora dell'ordine e chi ha effettuato l'ordine
					\item data/ora presa in carico dall'operaio stampatore dell'ordine e chi lo ha
					preso in carico
					\item data/ora spedizione (affidamento al corriere)
					\item SERVIZI: data/ora di consegna dell'ordine
					\item SERVIZI: registro dei contatti con dati dell'operaio e dell'utente che ha
					effettuato l'ordine
				\end{itemize}
		\end{itemize}
	\item \textbf{Operaio Stampatore}\\
	L'operaio stampatore si occupa anche del confezionamento della maglietta da affidare al corriere.
		\begin{itemize}
			\item pagina degli ordini non ancora gestiti da cui pu� scegliere un ordine da prendere 
			in carico (pu� per� avere pi� di un ordine in carico)
			\item pagina degli ordini che ha in carico (operazione di: chiusura dell'ordine e di evasione
			ordine (affido al corriere))\\
			SERVIZI: ID spedizione dal corriere 
			\item ? annullamento dell'ordine da parte dell'operaio per problemi ? [vedere se mettere 
			anche questo]
		\end{itemize}
	\item \textbf{Utente Registrato}
		\begin{itemize}
			\item pagina del catalogo dei prodotti e pagine che visualizza anche l'utente non registrato
			\item pagina creazione/personalizzazione della maglietta (? multiordine ? [vedere se mettere
			questo]
			\item pagina della lista degli ordini effettuati e stato ordine con contenuti
			\item pagina di annullamento dell'ordine fino a che non � stato preso in carico dall'operaio
			\item SERVIZI: modulo di contatto utente - operaio stampatore anonimo l'uno con l'altro,
			ma l'amministratore pu� vedere, per eventuali problemi
		\end{itemize}
	\item \textbf{Utente: Home Page}
		\begin{itemize}
			\item pagina del catalogo dei prodotti
			\item pagina per la possibilit� di registrazione
			\item pagina di presentazione dell'azienda: chi siamo, dove siamo, cosa facciamo
			\item pagina di informazioni sulle spedizioni genericamente: tempi di spedizioni, costi 
			di spedizione
		\end{itemize}
\end{itemize}

Il pagamento delle magliette avviene solamente Cash al corriere.

Poi vediamo cosa aggiugere di altro in base a cosa ci dice il Profe.

\end{document}
